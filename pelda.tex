\documentclass{article}% dokumentum fajtája (nem hagyható ki)

\usepackage{showframe}% az oldal részeit látni engedi
\usepackage{amsmath,mathtools}% matematikai dolgokhoz
\usepackage{amssymb}% bizonyos matematikai szimbólumohoz

\title{Példa}
\author{Búcsú Ákos}
\date{}

\begin{document}
	\maketitle% a cím megcsinálása

	\section{Másodfokú egyenlet}
	Egy másodfokú egyenlet általános alakja, ha $a,b,c\in\mathbb{R}$, $a\neq 0$, $x$ pedig valós változó:
	\begin{equation}
		ax^2+bx+c=0
	\end{equation}
	Ekkor
	\begin{gather}
		ax^2+bx+c=0\Big/:a\\
		%
		x^2+\frac{b}{a}x+\frac{c}{a}=0\\
		%
		\underbrace{\left(x+\frac{b}{2a}\right)^2-
		\left(\frac{b}{2a}\right)^2+\frac{c}{a}}_{\text{teljes négyzet}}=
		\left(x+\frac{b}{2a}\right)^2-
		\left(\frac{b^2-4ac}{4a^2}\right)=0\\
		%
		\left(x+\frac{b}{2a}+\sqrt{\frac{b^2-4ac}{4a^2}}\right)
		\left(x+\frac{b}{2a}-\sqrt{\frac{b^2-4ac}{4a^2}}\right)=0\\
		%
		\left(x+\frac{b+\sqrt{b^2-4ac}}{2a}\right)
		\left(x+\frac{b-\sqrt{b^2-4ac}}{2a}\right)=0
	\end{gather}
	Innen
	\begin{align*}% a csillag számozatlan sorokat eredményez
		x_1&=\frac{-b+\sqrt{b^2-4ac}}{2a}\\
		x_2&=\frac{-b-\sqrt{b^2-4ac}}{2a}
	\end{align*}
	Egy képletben összefoglalva:
	\begin{equation*}% a csillag az equation-re is működik
		x_{1,2}=\frac{-b\pm\sqrt{b^2-4ac}}{2a}
	\end{equation*}
\end{document}
