\documentclass[magyar]{article}
%
%%for page layout
%\usepackage{geometry}
% \geometry{
% a4paper,
% total={170mm,257mm},
% left=20mm,
% top=20mm,
% }
%
\input{preamble_latex_tutorial.tex}

\title{Latex Tutorial}
\author{Búcsú Ákos}
\date{}

\begin{document}
\maketitle

\section{Bevezetés}

Ezt a szöveget kiindulási alapként lehet csak használni, korántsem egész, vagy nagyon részletes kézikönyv. Azt tanácsolom, hogy az internet szorgos forgatásával sajátítsuk el a latexot mégpedig olyan dokumentumok írásával, ami hasznos (tehát jegyzetek, képletgyűjtemények stb.).

Kezdésnek ajánlan tudom az online \href{https://www.overleaf.com/sso-login}{Overleaf-et}, ami lehetővé teszi, hogy a különböző Latex-környezetek telepítése nélkül rövidebb dokumentumokat meg lehessen írni. Tanácsos az ebben a részben levő példakódot bemásolni oda és a kód módosításával kipróbálni (ami sokszor azt jelenti, hogy teljes összecsinálásra késztetni) a különböző funkciókat.

Hasznos linkek még:
\begin{enumerate}
	\item \url{https://en.wikibooks.org/wiki/LaTeX}
	\item \url{https://tikz.dev/}
	\item \url{https://latex-tutorial.com/}
\end{enumerate}
\pagebreak

\section{Példakód}

\lstinputlisting[language=tex,style=latex-dark]{pelda.tex}

\pagebreak

\section{Szövegtörzs}
A kód lényegi része a \verb|\begin{document}| és a \verb|\end{document}| között van. Az elnevezésből nyilvánvaló, hogy ezek jelzik a dokument elejét és végét, a közöttük levő részt én \textbf{szövegtörzsnek} fogom nevezni.

A szöveg bekezdésekre (paragraphs) tagolódik, ami a \verb|.tex| fájlban egy üres sorral érhető el. Tehát az üres sorok latexban jelenthetnek különbséget. Például a 34. sorban az "Innen" szó előtt azért nincs egy üres sor, hogy az "Innen" ne kerüljön új bekezdésbe. Ilyenkor érdemes egy \verb|%|-jellel helyettesíteni az üres sort, hiszen az utánna levő a kód, (esetünkben az üres sor) fordításakor nem lesz figyelembe véve

A \verb|\maketitle|-t majd a preamble résznél tárgyaljuk.

A \verb|\| azt jelenti, hogy "itt történni fog valami". Alapvetően három különböző eset van:
\begin{enumerate}
	\item \verb|\valami1|
	\item \verb|\valami2{valami}|
	\item \verb|\begin{valami3}| ... \verb|\end{valami3}|
\end{enumerate}

Az első eset például a \verb|\section{sectionnév}| ami egy új sectiont kezd, mi a következő sectionig tart. A sectionök számozva vannak és a szám után a sectionnév szerepel. Általánosságban a \verb|\valami| kapcsolóként működik. Ha például egy szövegrészt vastagra akarunk változtatni, akkor ezt a \verb|\bf|-fel lehet megtenni. Azonban ha csak egy szót akarunk vastagon, akkor arra a \verb|\textbf{vastagszó}|-t lehet használni.

Tehát a második esetnél a kapcsos zárójelekkel lehet meghatározni, hogy a \verb\valami\ hol hasson.

A harmadik eset egy környezetet (environment) határoz meg. A környezetben \textit{más szabályok érvényesek}, mint a sima szövegben. Nekünk különösen fontosak a matematikai környezetek, amiknél rögtön van egy kivétel, ez pedig a dollárjel, ami hasonlóan a \verb|\|-hez azt jelzi, hogy valami fog történni, de ennél konkrétabban azt jelzi, hogy valami matematikai dolog fog történni.

	\subsection{Matematikai környezetek}
	A dollárjel arra való, hogy a sima szövegbe matematikai kifejezéseket tudjunk írni.
	\lstinputlisting[language=tex,style=latex-dark,linerange={15-15}]{pelda.tex}
	A többi matematikai környezet mind egy új sorba helyezi a matematikai kifejezéseket, azonban dollárjelek közötti matematikai kifejezések a sima szövegben maradnak. A \verb|\in| (eleme), \verb|\neq| (nem egyenlő, \textbf{n}ot \textbf{e}qual) egy példa rögtön arra, hogy egy környezetben más szabályok érvényesek, mert míg a \verb|\neq| és \verb|\in| \verb|\valami| alakúak, mégsem kapcsolóként viselkednek, hanem matematikai karakterek kódjaként.

	Az \verb|\equation|,\verb|\gather|,\verb|\align| mind olyan matematikai környezetek, ahol nincs szükség dollárjelekre, mert magátó értetődően matematikai kifejezések kerülnek bele, tehát ha sima szöveget szeretnénk írni, akkor azt kell külön jelezni, erre láthatunk példát:
	\lstinputlisting[language=tex,style=latex-dark,linerange={20-35}]{pelda.tex}
	A \verb|\gather|-rel több egyenletet lehet egymás alá írni úgy, hogy nem lehet vízszintesen egymáshoz igazítani őket (az \verb|\align|-nál látni fogjuk, hogy hogyan lehet az egyenleteket igazítani). Mondtuk, hogy a matematikai szövegben külön kell jelezni, hogy rendes szöveget akarunk írni, ez látható a fenti kódrészlet 6. sorában, ahol a \verb|\text{rendes szöveg}| jelöli a rendes szöveget a matematikai környezetben.

	Ugyancsak a fenti részletben láthatjuk a \verb|_| és \verb|^| karaktereket, amik az alsó- és a felső indexet jelölik. Ha az indexbe több karaktert szeretnénk írni (a \verb|\|-lel kezdődő karakter is -- noha a kódban több karakter -- egy karakternek számít), ahogy azt láthatjuk az alábbi kódrészletben:
	\lstinputlisting[language=tex,style=latex-dark,linerange={42-44}]{pelda.tex}

	Most visszatérünk erre kódrészletre:
	\lstinputlisting[language=tex,style=latex-dark,linerange={20-35}]{pelda.tex}
	A különböző sorban levő egyenleteket \verb|\\| választja el egymástól, a \verb|%| pedig nem szükséges, csupán az üres sort jelöli (ha tényleges üres sort írnánk, akkor hibát kapnánk, ahogy ezzel szembefutottam a mostani dokumentum írása során).

	A \verb|\Big/|, a \verb|\left(| és a \verb|\right)| különleges karakterek, a \verb|Big| méretet határoz meg, a \verb|\left| és a \verb|\right| pedig a képlet méretéhez alkalmazkodó úgynevezett paired delimitereket (pl. \verb.()[]{}||.).

	A \verb|\frac{számláló}{nevező}| egy kétparaméteres \verb|\|-kifejezés és szép törtet jelöl.

	A \verb|\gather| minden sora egy sorszámot kap, ahogy minden többsoros matematikai környezet (multiline environment).
\end{document}
